\section{Realisierung}

... Wie wurden die beiden Ideen umgesetzt ...

\subsection{Architekturbeschreibung}

Die Anwendung beruht auf dem Client-Server-Prinzip. Dabei stellt der Client lediglich die Oberfläche zur Interaktion mit dem Anwender dar. Außer der notwendigen UI Logik ist die gesamte Geschäftslogik auf einen dedizierten (Business-)Server ausgelagert. Dieser stellt ebenfalls die Datenbank bereit. Mittels AJAX und REST werden dynamische Daten vom Client beim Server angefragt.\\

... Beschreibung (mit Schema) der Softwarearchitektur ...
... Tier2/Tier3 ...

\subsection{Clientkomponente}
\subsubsection{Umsetzung ServiceWorker}

... Umsetzung mittels ServiceWorker ...

\subsubsection{Umsetzung mittels nativem Android Service}

... Umsetzung mittels eigenem nativen Service ...


\subsection{Serverkomponente}
\subsubsection{API Beschreibung}


\begin{table}[h]
\centering
\begin{tabular}{l | c | l }
    \textbf{Route} & \textbf{HTTP-Methode} & \textbf{Beschreibung} \\
    \hline\hline
    /api/signup & POST & Registriert einen neuen Benutzer \\
    /api/authenticate & POST & Authentifiziert einen Benutzer \\
    \hline
    /api/tasks & GET & Gibt alle Aufgaben zurück \\
    /api/tasks & POST & Legt eine neue Aufgabe an \\
    /api/tasks/:taskId & GET & Gibt eine einzelne Aufgabe zurück \\
    /api/tasks/:taskId & PUT & Aktualisiert eine einzelne Aufgabe \\
    /api/tasks/:taskId & DELETE & Löscht eine einzelne Aufgabe \\
\end{tabular}
\caption{Übersicht API Routen}
\label{tbl_api-routes}
\end{table}
