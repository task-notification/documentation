\chapter{Grundlagen}

\section{Serviceworker}
\label{sec_grundladge_serviceworker}

Ein Service Worker ist eine W3C-Standard-Webtechnik bei der JavaScript-Code im Hintergrund von Web-Browsern ausgeführt wird. Mit Hilfe von Service Worker ist es möglich, essentielle Funktionalitäten wie Caching zur Offline"=Verwendbarkeit (z.B. bei Ausfall der Internetverbindung) von Web-Anwendungen, Aktualisierungen von Inhalten im Hintergrund, aber auch die von nativen Apps bekannten Push"=Benachrichtigungen(Push"=Notifications) zu ermöglichen. Dies findet alles im Hintergrund des Browsers statt und macht somit eine Installation von Software oder Software-Diensten unnötig.

Der Service Worker kann als Proxy fungieren und zum anderen vom Server gesendete Benachrichtigungen, selbst dann empfangen, wenn gerade keine Web-Page der entsprechenden Domain / Web-App geöffnet ist. 

\section{Web Push API}
\label{sec_grundlagen_web-push}
    
Bei Web Push handelt es sich um eine Erweiterung des bekannten Service-Worker-Standards. Solange der Browser geöffnet ist, können Benachrichtigungen von Webseiten empfangen werden, selbst wenn der eigentliche Tab nicht geöffnet ist. So kann man E-Mail-Tab schließen und trotzdem über eingehende Mails informiert werden. Da keine zusätzlichen Apps oder Text-Nachrichten für direkte Notifications nötig sind, ergibt sich ein großer Vorteil für Speichernutzung, Performance und Akkulaufzeit von Mobilgeräten. \\
Web Push benötigt genauso wie die Standortfreigabe oder der Kamerazugriff eine (jederzeit wiederrufbare) Berechtigung, bevor eine Webseite auf Push-Events reagieren und Notifications anzeigen kann. 

Durch eine ständige Verbindung zu einem Push Service in unserem Fall „Firebase Cloud Messaging“ , der als zentrale Schaltstelle für Nachrichten fungiert, werden Web-Push-Benachrichtigungen ermöglicht. Ursprünglich betrieb jeder Browser-Anbieter einen eigenen Push"=Service zum Schutz der Privatsphäre. Erst kürzlich wurden aber GCM (Google Cloud Messaging Push Service von Google) und Firebase (Mozilla Firefox Push Service) zu Firebase Cloud Messaging zusammengelegt. \\
Dabei erhält jede Webseite einen anderen, anonymen Web Push Identifier zur Verhinderung von seitenübergreifenden Zuordnungen. Zudem müssen die Nutzerdaten über ein Public-Key-Verfahren verschlüsselt werden. Der Service Worker meldet sich nur beim Push-Dienst an, wenn der User die notwendigen Push"=Berechtigungen erteilt hat. 