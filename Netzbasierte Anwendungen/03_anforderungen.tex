\chapter{Anforderungen}
\label{chp_anforderungen}

\section{allgemeine Beschreibung der Applikation}
\label{sec_anfroderungen_allgemeine-beschreibung}

Nach erfolgreicher Registrierung und Anmeldung kann der Benutzer Aufgaben anlegen, bearbeiten, anzeigen und löschen. Weiterhin gibt es eine Kontaktliste, in welcher alle Kontakte angezeigt werden, die ebenfalls für die Anwendung registriert sind und zu persönlichen Kontakten hinzugefügt wurden. Aufgaben können mit persönlichen Kontakten geteilt werden. 

Über Änderungen an Gruppen oder Aufgaben wird der Benutzer über PUSH"=Benachrichtigungen informiert. Wenn einer Aufgabe ein Benachrichtigungszeitpunkt angegeben wurde, wird ebenfalls eine PUSH"=Notification angezeigt sobald die Aufgabe terminiert.

\todo{Die Beispielanwendung als Einsatzszenario beschreiben\ldots}

\newpage
\section{funktionale Anforderungen}
\label{sec_anforderungen_funktionale-anforderungen}

Im Rahmen dieser Dokumentation werden unter funktionalen Anforderungen diejenigen verstanden, welche zur direkten Zielerfüllung beitragen (vgl. \ref{sec_einleitung_motivation-ziele}).

\todo{weiter ausführen, was damit gemeint ist\ldots}

\paragraph{[FA-1] Single Page (Mobil) Application.} Die Webanwendung wird als hybride Single Page Application entwickelt. Der Sitzungszustand (\glqq Application State\grqq) wird auf dem Client gespeichert und im Hintergrund mit dem Applicationserver synchronisiert. Die Präsentationslogik wird in der \glqq Boot"=Phase\grqq{} einmal vom Client geladen und Inhalte zur Laufzeit dynamisch angepasst. Während der Navigation wird der Präsentationsfluss im Client nicht angehalten bzw. durch \glqq neu Laden\grqq{} unterbrochen.

\paragraph{[FA-2] Offlinefähigkeit.} Die Benutzung der Webanwendung soll nicht ausschließlich bei bestehender Internetverbindung, sondern ebenfalls Offline reibungslos möglich sein. Dazu bietet die progressive Webanwendung Mechanismen zum Vorhalten der statischen Ressourcen und des Sitzungszustands der Anwendung (\glqq Application State\grqq) im Offlinemodus. Benutzer werden über ggf. eingeschränkte Funktionalitäten informiert, während keine aktive Internetverbindung vorhanden ist.

\paragraph{[FA-3] Push-Benachrichtigungen.} Benutzer der Webanwendung werden unabhängig vom verwendeten Endgerät über bestimmte Ereignisse mit Hilfe von Push"=Benachrichtigungen informiert. Diese Ereignisse werden vom Applicationserver ausgelöst/verarbeitet und dieser initiiert Push"=Benachrichtigungen beim Client. 

\paragraph{[FA-4] Schnittstelle für Kommunikation mit Applicationserver.} Der API Server unterstützt folgende Anforderungen um die Funktionalitäten einer RESTful"=Schnittstelle zu erfüllen:

\begin{itemize}
\setlength{\itemsep}{0pt}%
\setlength{\parskip}{0pt}%
\item Bereitstellung von CRUD\footnote{\textit{CRUD: create, read, update, delete}}-Funktionalität für Entities
\item Aufruf von Ressourcen über eindeutige und einfache URLs (z.B. \url{https://example.de/api/task/} und \url{https://example.de/api/task/:taskId}) 
\item Verwendung der standardisierten HTTP-Methoden (GET, POST, PUT und DELETE) 
\item Rückgabe im JSON-Format
\item alle Requests werden auf der Konsole ausgegeben
\end{itemize}  

\paragraph{[FA-5] Gesicherter Zugriff auf API.} Der Zugriff auf die API ist nur für authentifizierte Benutzer möglich. Bei Ressourcenanforderung muss ein Token im \code{Authorization"=Header} oder \code{Request"=Body} übergeben werden. Im Fehlerfall sollen aussagekräftige und passende HTTP"=Fehler zurückgegeben werden.

\section{nicht-funktionale Anforderungen}
\label{sec_anforderungen_nicht-funktionale-anforderungen}

Im Rahmen dieser Dokumentation werden unter nicht-funktionalen Anforderungen diejenigen verstanden, welche nicht zur direkten Zielerfüllung beitragen (vgl. \ref{sec_einleitung_motivation-ziele}).

\todo{weiter ausführen, was damit gemeint ist\ldots}

\paragraph{[NFA-1] Look \& Feel einer nativen Android"=App.} Die Website soll sich optisch an den \glqq Material Design\grqq "=Richtlinien für Android"=Applikationen orientieren. Auf mobilen Endgeräten soll die Touch"=Unterstützung genauso gewährleistet sein, wie die für Smartphones typischen Wisch"=Gesten. Das Framework \glqq JQuery"=Mobile\grqq{} mit zugehörigem UI"=Framework \glqq JQuery"=Mobile"=UI\grqq{} kann als JQuery"=Erweiterung zu diesem Zweck eingesetzt werden.

\paragraph{[NFA-2] Benutzerauthentifizierung.} Benutzer können sich für die Nutzung der Anwendung Registrieren und anschließend Anmelden. Für die Registrierung ist ein eindeutiger Benutzername mit Angabe einer E-Mail Adresse sowie ein Passwort notwendig.

\paragraph{[NFA-3] Kontaktliste.} Benutzer können sich untereinander mittels Benutzername bzw. E"~Mail Adresse zur persönlichen Kontaktliste hinzufügen. Benutzer werden über Freundschaftsanfragen benachrichtigt und können diese bestätigen oder ablehnen.

\paragraph{[NFA-4] Aufgaben anlegen, bearbeiten und löschen.} Ein authentifizierter Benutzer kann Aufgaben anlegen und anschließend Bearbeiten oder Löschen. Eine Aufgabe muss einen Titel besitzen. Optional kann eine Beschreibung und ein Fälligkeitsdatum hinterlegt werden.

\paragraph{[NFA-5] Aufgaben teilen.} Aufgaben können mit mehreren Benutzern geteilt werden. Benutzer werden über die Einladung zu einer Aufgabe informiert. Nachdem die Einladung bestätigt wurde, wird ein Benutzer über Änderungen an der Aufgabe benachrichtigt.

\paragraph{[NFA-6] Ereignisse für Benachrichtigungen.} Benutzer werden über Freundschaftsanfragen und Änderungen an Aufgaben informiert, in denen sie involviert sind. Benachrichtigungen werden nur angezeigt, wenn die Webanwendung nicht aktiv ist. Die Website gilt als \glqq aktiv\grqq, wenn sie auf dem Bildschirm angezeigt wird. \\
Folgende Ereignisse lösen eine Benachrichtigung aus:

\begin{itemize}
\setlength{\itemsep}{0pt}%
\setlength{\parskip}{0pt}%
\item Freundschaftsanfrage wurde von einem anderen Benutzer gestellt
\item Freundschaftsanfrage wurde durch einen anderen Benutzer bestätigt/abgelehnt
\item ein anderer Benutzer hat die eigene Freundschaftsanfrage bestätigt/abgelehnt
\item Einladung zu einer Aufgabe durch einen anderen Benutzer
\item Bestätigung/Ablehnung durch einen Benutzer auf eine Einladung zu einer Aufgabe
\item Änderungen an einer Aufgabe, an welcher der Benutzer beteiligt ist
\end{itemize}
