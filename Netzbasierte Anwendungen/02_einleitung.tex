\chapter{Einleitung}

Diese Dokumentation entstand im Rahmen der Profilierung \glqq Netzbasierte Anwendungen\grqq{} im Wintersemester 2016/17 an der Hochschule für Telekommunikation Leipzig (HfTL).

\todo{Wer hat welches Kapitel geschrieben\ldots?}

\section{Motivation und Ziele}
\label{sec_einleitung_motivation-ziele}

Die Anzahl von mobilen Applikationen wächst heutzutage immer schneller an. Diese Feststellung gilt sowohl für die sog. Store-Apps (die öffentlich den Kunden zur Verfügung gestellt werden) als auch für Enterprise Applikationen, die in dem Backend des Unternehmens integriert werden. Viele von diesen Anwendungen funktionieren aber nur dann, wenn eine Verbindung mit dem Internet besteht.

Manchmal, selbst im Fall von kurzen Verbindungsabbrüchen, ist die weitere Arbeit mit der Anwendung unmöglich. Dies kann allerdings in manchen Geschäftsszenarien nicht akzeptabel sein. Um die Benutzererfahrung zu verbessern und dieses Problem zu lösen, bieten viele Anwendungen Offlinefähigkeit an. Wenn man sich diese mobilen Anwendungen genauer anschaut, scheint die Verwendung des Begriffs der Offlinefähigkeit inkonsistent zu sein. Es gibt keine einheitliche Implementierung von Offlinefähigkeit. 


\todo{Einleitung schreiben\ldots}
... Einleitung moderne webtechnologien --> webapps statt nativen Apps ... \\
... Beschreibung der Aufgabe/des Problems ...\\

... Versuch der Lösungsfindung/Kurzbeschreibung Projekt ... \\

