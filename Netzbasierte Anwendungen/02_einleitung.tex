\chapter{Einleitung}

Fast jeder Mensch benutzt Sie, installiert Sie und wendet sie an. Die Rede ist von nativen App`s, hierbei handelt es sich um installierbare und ausführbare Applikationen, die auf Mobilgeräten, Desktop PC`s oder aber auf Tablets ihren Dienst verrichten und uns das Leben vereinfachen oder aber auch einfach nur Informieren und zu Kommunikation genutzt werden. Eine weitere Technologie die mindestens genauso Anklang findet, sind herkömmliche Webseiten von Anbietern wie z.B. Reiseunternehmen, Firmen mit einer Internetpräsenz und viele weitere.

In den letzten Jahren kam der Begriff der \glqq Progressive Web Apps\grqq{} auf, doch was bedeutet dieser und was für Vorteile kann uns diese neue in den Anfängen steckende Technologie bieten? Haben \glqq Progressive Web Apps\grqq{} die Möglichkeit, herkömmliche Technologien wie traditionelle native Apps zu verdrängen und was macht Sie aktuell so beliebt?

Eine Progressive Web App verbindet die Vorteile zweier Welten; HTML/CSS"=basierte Webseiten und native Apps. Sie ist wie eine normale Webseite in einem Browser aufrufbar und schafft aber ein authentisches App"=Erlebnis gegenüber dem User. Das authentische Gefühl, wird dem User durch Techniken, die auch in herkömmlichen, traditionellen Apps zum Tragen kommen, wie etwa Push"=Nachrichten, die über auftretende Ereignisse informieren oder schnelle Ladezeiten und generelle Nutzung der App vermittelt, selbst bei schlechter Internetverbindung. Die Offline-Bedienbarkeit und das Echtzeit-Erlebnis ermöglichen eine permanente Verfügbarkeit der Webseite.  
Im Gegensatz zu traditionellen nativen Apps für Tablet, Smartphone und Desktop PC können progressive Web Apps ohne eine Installation auf den Endgeräten auskommen. 

Durch die Implementierung der Technologie \glqq Service Worker\grqq{} von großen Browserherstellern wie Google und Firefox, wird es überhaupt erst möglich Push"=Nachrichten zu empfangen und zu verarbeiten.

Doch hält diese Technologie, was sie verspricht und mit welchen Techniken kann das umgesetzt werden? Deshalb wurde dieses Projekt ins Leben gerufen, um genau diese neue Technologie, der \glqq Progressiven Web App\grqq{} zu untersuchen und am Beispiel einer simplen Anwendung zu verdeutlichen.

Im Rahmen der Profilierung \glqq netzbasierte Anwendungen\grqq{} an der Hochschule für Telekommunikation Leipzig (HfTL) betrachtet diese Dokumentation die Möglichkeiten der Offlinefähigkeit und verarbeitung von Pish"=Benachrichtigungen in progressiven Web Apps.
