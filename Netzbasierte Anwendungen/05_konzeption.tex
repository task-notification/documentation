\chapter{Konzeption}
\label{sec_konzeption}

\section{Caching der statische Resourcen}

Tabelle zeigt die füng grundsätzlich möglichen Strategien für das Caching.


\begin{table}[h]
\centering
\begin{tabularx}{\textwidth}{| l | X | }
    \hline
    \textbf{Bezeichnung} & \textbf{Beschreibung} \\
    \hline
    networkOnly & Ressourcen werden nur aus Netzwerk geholt \\
    \hline    
    cacheOnly & Ressourcen werden immer aus Cache geladen \\
    \hline
    fastest & Versucht von beiden Quellen zu laden und Antwortet mit schnellerem Response \\
    \hline
    networkFirst & Versucht zuerst aus dem Netzwerk zu laden und schaut in den Cache, wenn dies fehlschlägt \\
    \hline
    cacheFirst & Bezieht Ressourcen direkt aus dem Cache, fragt jedoch auch beim Netzwerk nach und aktualisiert bei Erfolg die Ressourcen im Cache \\
    \hline
\end{tabularx}
\caption{Übersicht Caching Strategien}
\label{tbl_konzeption_caching-strategien}
\end{table}


\section{Web Push}
\label{subsubsec_konzeption_serviceworker_push-api}

\begin{figure}[htp] \centering{
\centering
\includegraphics[width=0.9\textwidth]{images/architektur_serviceworker_push.pdf}}
\caption{Push mittels Serviceworker (in Anlehnung an MozillaWiki \cite{MOZ_WIKI})}
\quelle\url{https://wiki.mozilla.org/File:PushNotificationsHighLevel.png}
\label{image_architektur-serviceworker-push}
\end{figure}  

... Beschreibung (mit Schema) der Softwarearchitektur ...


\section{Architekturbeschreibung}
\label{sec_konzeption_serviceworker_architektur}

Die Anwendung beruht auf dem Client-Server-Prinzip. Dabei stellt der Client lediglich die Oberfläche zur Interaktion mit dem Anwender dar. Außer der notwendigen UI- und Serviceworker-Logik ist die gesamte Geschäftslogik auf einen Business-Server (Applicationserver) ausgelagert. Die zentrale Datenbank sowie die statischen Ressourcen zur Darstellung des Client werden ebenfalls vom Applicationserver bereitgestellt. Für die Kommunikation steht eine RESTful-Schnittstelle zur Verfügung.

\begin{figure}[htp] \centering{
\includegraphics[width=0.9\textwidth]{images/architektur_serviceworker.pdf}}
\caption{Archtikturbeschreibung - Umsetzung mit Serviceworker}
\label{image_architektur-serviceworker-push}
\end{figure} 

\newpage
\section{Applicationserver}
\label{sec_konzeption_applicationserver}

\subsection{Datenbank}



\subsection{REST-API}

Zur Bereitstellung von CRUD-Funktionalitäten über standardisierte HTTP-Methoden (vgl. \ref{subsec_anforderungen_server} Anforderungen an Serverkomponente) wird dem Applicationserver eine RESTful-Schnittstelle hinzugefügt. Eine Übersicht über mögliche API-Routen mit entsprechender HTTP-Methode ist in Tabelle \ref{tbl_konzeption_rest} dargestellt. \\

\begin{table}[h]
\centering
\begin{tabular}{l | c | l }
    \textbf{Route} & \textbf{HTTP-Methode} & \textbf{Beschreibung} \\
    \hline\hline
    /api/signup & POST & Registriert einen neuen Benutzer \\
    /api/authenticate & POST & Authentifiziert einen Benutzer \\
    \hline
    /api/tasks & GET & Gibt alle Aufgaben zurück \\
    /api/tasks & POST & Legt eine neue Aufgabe an \\
    /api/tasks/:taskId & GET & Gibt eine einzelne Aufgabe zurück \\
    /api/tasks/:taskId & PUT & Aktualisiert eine einzelne Aufgabe \\
    /api/tasks/:taskId & DELETE & Löscht eine einzelne Aufgabe \\
    \hline
    /push/devices & GET & Gibt alle registrierten Geräte zurück \\
    /push/devices & POST & Registriert ein neues Gerät \\
\end{tabular}
\caption{Übersicht API Routen}
\label{tbl_konzeption_rest}
\end{table}

\subsubsection{Authentifizierung und Autorisierung} 

Für den Zugriff auf die CRUD-Methoden ist eine Benutzerauthentifizierung und Autorisierung notwendig. Dazu wird das Token-Verfahren verwendet.

\newpage
\section{Client-Oberfläche}
\label{sec_konzeption_client-ui}


... Mockup ... \\
... Beschreibung des UI ...\\

Um das \glqq{}Look and Feel\grqq{} einer nativen App zu erreichen wird das UI-Framework \textbf{nativeDroid2} verwendet. \\

\newpage
\section{Datenmodel}
\label{sec_konzeption_datamodel}

\subsection{Datenbankschema}

\begin{figure}[htp] \centering{
\includegraphics[width=0.9\textwidth]{images/model.png}}
\caption{Datenmodell}
\label{image_konzeption_datenmodell}
\end{figure} 


\subsection{Bereitstellung der Daten}

\newpage
