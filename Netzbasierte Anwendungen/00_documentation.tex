%----------------------------------------------------------------------------------------
%	PACKAGES AND OTHER DOCUMENT CONFIGURATIONS
%----------------------------------------------------------------------------------------

\documentclass[fontsize=11pt, paper=a4, headinclude, twoside=false, parskip=half+, pagesize=auto, numbers=noenddot, open=right]{scrreprt} %{article}

\usepackage{times}
\fontfamily{ptm}\selectfont
\renewcommand*\rmdefault{ptm}

\usepackage[left= 2cm,right = 2cm, bottom = 2 cm]{geometry}
\usepackage[onehalfspacing]{setspace}
\usepackage[german]{babel}
\usepackage[utf8x]{inputenc}
\usepackage{amsmath}
\usepackage{graphicx}
\usepackage{epstopdf}
\usepackage[colorinlistoftodos]{todonotes}
\graphicspath{{images/}}
\usepackage{fancyhdr}
\usepackage{caption}
\usepackage{pdfpages}
\usepackage[export]{adjustbox}
\usepackage[T1]{fontenc}            % Ermöglicht die automatische Trennung von Worten mit Umlauten
\usepackage{amssymb}
\usepackage{ltxtable}
\usepackage{booktabs-de}
\usepackage{caption}
\usepackage{adjustbox}
\usepackage{wrapfig}


% Dokumentinformationen
\usepackage[
	pdftitle={Dokumentation Service Worker},
	pdfsubject={Projektdokumentation},
	pdfauthor={David Howown, Michael Müller},
	pdftex=true, 
	colorlinks=true,
 	breaklinks=true,
	citecolor=black,
	linkcolor=black,	
	menucolor=black,	
	urlcolor=black
]{hyperref}

% fussnote ganz am Ende der Seite
\usepackage[bottom]{footmisc}

% code snippet
\usepackage{listings}
\usepackage{color}

\definecolor{dkgreen}{rgb}{0,0.6,0}
\definecolor{gray}{rgb}{0.5,0.5,0.5}
\definecolor{mauve}{rgb}{0.58,0,0.82}

\lstset{frame=tb,
  language=Java,
  aboveskip=3mm,
  belowskip=3mm,
  captionpos=b,
  breakatwhitespace=true,
  basicstyle=\footnotesize,
  showstringspaces=false,
  columns=flexible,
  basicstyle={\footnotesize\ttfamily},
  numberstyle=\tiny\color{gray},
  keywordstyle=\color{blue},
  commentstyle=\color{dkgreen},
  stringstyle=\color{mauve},
  breaklines=true,
  breakatwhitespace=true,
  tabsize=2,
  numbers=left,
  stepnumber=1,    
  firstnumber=1,
  numberfirstline=true
}

%  Bibliographie
\usepackage{bibgerm} % Umlaute in BibTeX
\usepackage{csquotes}

% nicht einrücken nach Absatz
\setlength{\parindent}{0pt}

\def\code#1{\texttt{{\small #1}}}
\usepackage{etoolbox}
\makeatletter
\gdef\tshortstack{\@ifnextchar[\@tshortstack{\@tshortstack[c]}}
\let\@tshortstack\@shortstack
\patchcmd\@tshortstack\vbox\vtop{}{}
\makeatother


\begin{document}

\pagestyle{empty}
\begin{titlepage}

\newcommand{\HRule}{\rule{\linewidth}{0.5mm}} % Defines a new command for the horizontal lines, change thickness here

\center % Center everything on the page
 
%----------------------------------------------------------------------------------------
%	HEADING SECTIONS
%----------------------------------------------------------------------------------------

\LARGE Hochschule für Telekommunikation Leipzig (HfTL)\\[1.2cm] % Name of your university/college
\textsc{\Large Profilierung Netzbasierte Anwendungen}\\[0.5cm] % Major heading such as course name
\textsc{\Large Profilierung Mobile Applikationen}\\[0.5cm] % Major heading such as course name
\textsc{\large Softwaredokumentation}\\[0.5cm] % Minor heading such as course title

%----------------------------------------------------------------------------------------
%	TITLE SECTION
%----------------------------------------------------------------------------------------

\HRule \\[0.4cm]
{ \LARGE \bfseries TaskY - Cache und Notifications in mobilen Webanwendungen}\\ % Title of your document
\HRule \\[1cm]
 
%----------------------------------------------------------------------------------------
%	AUTHOR SECTION
%----------------------------------------------------------------------------------------

\Large David Howon (147102)\\[1cm] 

\Large Michael Müller (147105)\\[1cm] 

%----------------------------------------------------------------------------------------
%	DATE SECTION
%----------------------------------------------------------------------------------------

{\large Wintersemester 2016/17}\\[2cm] % Date, change the \today to a set date if you want to be precise

%----------------------------------------------------------------------------------------
%	LOGO SECTION
%----------------------------------------------------------------------------------------

\includegraphics[width=0.4\textwidth]{hftl_logo.eps}\\[1cm] % Include a department/university logo - this will require the graphicx package
 
%----------------------------------------------------------------------------------------

\vfill % Fill the rest of the page with whitespace

\end{titlepage}

% Seitenränder anpassen
\newgeometry{
    left = 2cm, right = 3.5cm, bottom = 2.5cm
}

\renewcommand*{\chapterheadstartvskip}{\vspace*{.4\baselineskip}}% Abstand 
\renewcommand*\chapterpagestyle{fancy}

% ============= Kopf- und Fußzeile =============
\pagestyle{fancy}
%
\lhead{Dokumentation Service Worker}
\chead{}
\rhead{\slshape \leftmark}
%%
\lfoot{
    \includegraphics[scale=0.6, valign=c]{hftl_logo_sm.png}
}
\cfoot{
%{\scriptsize David Howon (147102) und Michael Müller (147105)}
}
\rfoot{\textbf{\thepage}}
%%
\renewcommand{\headrulewidth}{0.4pt}
\renewcommand{\footrulewidth}{0.0pt}

% Abstand Fussnote und Fusszeile
\setlength{\footskip}{-0.5cm} 
\setlength{\skip\footins}{0.5cm} 
% Abstand zwischen Fussnoten
\setlength{\footnotesep}{0.3cm}

% \part im Inhaltsverzeichnis nicht nummerieren
\makeatletter
\let\partbackup\l@part
\renewcommand*\l@part[2]{\partbackup{#1}{}}

\newcommand*{\quelle}{%
  \footnotesize Quelle:
}

% spacing von Tabellen Zellen
\def\arraystretch{1.5}%  1 is the default, change whatever you need

%Seitennummerierung neu beginnen, Zahlen [arabic], röm.Zahlen [roman,Roman], Buchstaben [alph,Alph]
\pagenumbering{Roman}
\newpage
\pagestyle{fancy}
%Inhaltsverzeichnis
\tableofcontents

\newpage
%Seitennummerierung neu beginnen, Zahlen [arabic], röm.Zahlen [roman,Roman], Buchstaben [alph,Alph]
\pagenumbering{arabic}
% pagestyle für gesamtes Dokument aktivieren
\pagestyle{fancy}

\addtocontents{toc}{\protect\setcounter{tocdepth}{0}} % 0 nur > \section zeigen
\addtocontents{toc}{\protect\setcounter{tocdepth}{2}} % nur bis \subsubsection (Standard 3)

\section{Einleitung}

Diese Dokumentation entstand im Rahmen der Profilierungen  \glqq{}Mobile Applikationen\grqq{} und \glqq{}Netzbasierte Anwendungen\grqq{} im Wintersemester 2016/17 an der Hochschule für Telekommunikation Leipzig (HfTL). \\


... Einleitung moderne webtechnologien --> webapps statt nativen Apps ... 

... Beschreibung der Aufgabe/des Problems ...

... Versuch der Lösungsfindung/Kurzbeschreibung Projekt ...
\\

\begin{Large}
Entwicklung
\end{Large} \\

\begin{tabular}{ l l l }
  GitHub & & \href{https://github.com/task-notification}{https://github.com/task-notification}  \\
  Demo & WebApp & \href{https://tasky.atria.uberspace.de}{https://tasky.atria.uberspace.de/} \\
   & Business-Server & TBA \\
\end{tabular}
\newpage

\chapter{Grundlagen}

\section{Serviceworker}
\label{sec_grundladge_serviceworker}

Ein Service Worker ist eine W3C-Standard-Webtechnik bei der JavaScript-Code im Hintergrund von Web-Browsern ausgeführt wird. Mit Hilfe von Service Worker ist es möglich, essentielle Funktionalitäten wie Caching zur Offline"=Verwendbarkeit (z.B. bei Ausfall der Internetverbindung) von Web-Anwendungen, Aktualisierungen von Inhalten im Hintergrund, aber auch die von nativen Apps bekannten Push"=Benachrichtigungen(Push"=Notifications) zu ermöglichen. Dies findet alles im Hintergrund des Browsers statt und macht somit eine Installation von Software oder Software-Diensten unnötig.

Der Service Worker kann zum einen als Proxy fungieren und zum anderen vom Server gesendete Benachrichtigungen, selbst dann empfangen, wenn gerade keine Webseite der entsprechenden Domain geöffnet ist. 

\section{Web Push API}
\label{sec_grundlagen_web-push}
    
Bei Web Push handelt es sich um eine Erweiterung des bekannten Service Worker Standards. Solange der Browser geöffnet ist können Benachrichtigungen von Webseiten empfangen werden, selbst wenn der eigentliche Tab nicht geöffnet ist. So kann man zum Beispiel einen E"=Mail"=Tab schließen und trotzdem über eingehende Mails informiert werden. Da keine zusätzlichen Apps oder Text-Nachrichten für direkte Notifications nötig sind, ergibt sich ein großer Vorteil für Speichernutzung, Performance und Akkulaufzeit von Mobilgeräten. \\
Web Push benötigt genauso wie die Standortfreigabe oder der Kamerazugriff eine (jederzeit wiederrufbare) Berechtigung, bevor eine Webseite auf Push-Events reagieren und Notifications anzeigen kann. 

Durch eine ständige Verbindung zu einem Push Service in unserem Fall „Firebase Cloud Messaging“ , der als zentrale Schaltstelle für Nachrichten fungiert, werden Web-Push-Benachrichtigungen ermöglicht. Ursprünglich betrieb jeder Browser-Anbieter einen eigenen Push"=Service zum Schutz der Privatsphäre. Erst kürzlich wurden aber GCM (Google Cloud Messaging Push Service von Google) und Firebase (Mozilla Firefox Push Service) zu Firebase Cloud Messaging zusammengelegt. \\
Dabei erhält jede Webseite einen anderen, anonymen Web Push Identifier zur Verhinderung von seitenübergreifenden Zuordnungen. Zudem müssen die Nutzerdaten über ein Public-Key-Verfahren verschlüsselt werden. Der Service Worker meldet sich nur beim Push-Dienst an, wenn der User die notwendigen Push"=Berechtigungen erteilt hat. 
\newpage

\section{Anforderungen}

\subsection{funktionale Anforderungen}

Die WebApp soll für den \textbf{Mehrbenutzerbetrieb} ausgelegt werden. Ein Benutzer soll sich für die Nutzung \textbf{Registrieren} und anschließend am Portal \textbf{Anmelden} können. Es können eigene \textbf{Aufgabenlisten} angelegt, bearbeitet oder gelöscht werden. Weiterhin können Aufgabenlisten mit mehreren Benutzern (Kontakte bzw. Gruppen) geteilt werden. \\

Zu einer Aufgabenliste können einzelne Aufgaben zugeordnet werden. Eine Aufgabe muss mindestens aus einem Titel bestehen und kann mit einem Ort, einer Beschreibung, einer hinterlegten Checkliste, einem Zeitraum sowie einer Fälligkeit erweitert werden. \\

Der Benutzer soll mittels PUSH-Benachrichtigungen über aktuelle Änderungen an den Aufgabenlisten und Gruppen informiert werden. 

\subsection{nicht-funktionale Anforderungen}

- Look and Feel einer nativen Android App

- Single Page Application (SPA)

\newpage

\chapter{Konzeption}
\label{sec_konzeption}

\section{Offlinefähigkeit}
\label{sec_konzeption_offline}

\subsection{Caching statischer Ressourcen}
\label{subsec_konzept_caching-statische-ressourcen}

Während Webanwendungen einen Fehler anzeigen, sobald der Benutzer ohne aktive Internetverbindung versucht zu einer Seite zu navigieren, ist es in nativen Apps möglich sich weiter innerhalb der Anwendung zu bewegen. 

Eine hybride Webanwendung muss also die Möglichkeit haben, zu erkennen, ob eine Internetverbindung vorhanden ist oder nicht und entsprechend reagieren. Hier kommt die Service Worker API ins Spiel. Hauptaugenmerkt der Technologie ist die Bereitstellung einer optimalen Offline"=Benutzererfahrung.   

Wie in \todo{Hier muss die Referenz hin} beschrieben handelt es sich beim Service Worker um eine Art Proxy zwischen der Webanwendung und dem Browser. Dadurch ist es möglich, Responses von HTTP-Request aufzunehmen und anzupassen. Dies ist eine Schlüsselfunktion, um Offlinefähigkeit anbieten zu können.

\begin{table}[h]
\centering
\begin{tabularx}{\textwidth}{| l | X | }
    \hline
    \textbf{Bezeichnung} & \textbf{Beschreibung} \\
    \hline
    networkOnly & Ressourcen werden nur aus Netzwerk geholt \\
    \hline    
    cacheOnly & Ressourcen werden immer aus Cache geladen \\
    \hline
    fastest & Versucht von beiden Quellen zu laden und Antwortet mit schnellerem Response \\
    \hline
    networkFirst & Versucht zuerst aus dem Netzwerk zu laden und schaut in den Cache, wenn dies fehlschlägt \\
    \hline
    cacheFirst & Bezieht Ressourcen direkt aus dem Cache, fragt jedoch auch beim Netzwerk nach und aktualisiert bei Erfolg die Ressourcen im Cache \\
    \hline
\end{tabularx}
\caption{Übersicht Caching Strategien}
\label{tbl_konzeption_caching-strategien}
\end{table}

Tabelle \ref{tbl_konzeption_caching-strategien} zeigt die fünf grundsätzlich möglichen Strategien für das Caching von statischen Ressourcen, die mit Hilfe des Service Workers umgesetzt werden können. Damit die Benutzung der Anwendung auch ohne aktive Internetverbindung gewährleistet ist, müssen die Ressourcen ebenfalls bereitstehen, wenn das Gerät offline ist. Dadurch das erkannt werden kann, ob das Gerät vom Internet getrennt ist und dadurch anders auf HTTP-Requests reqgieren werden kann, ergibt sich die Möglichkeit, Ressourcen auszuliefern, die lokal gespeichert sind.

Für den vorliegenden An \code{cacheFirst}-Verfahren bietet sich für die vorligenden Anwendungsfall an. Die angeforderten Ressourcen werden direkt aus dem Cache geladen und anschließend wird versucht, ob dieses mit Ressourcen aus dem Internet aktualisiert werden können(vgl. Bild \ref{image_konzept_caching-strategie}. Dadurch wird die Seite unabhängig vom Onlinezustand bei Anforderung schnell geladen.

\begin{figure}[htp] \centering{
\includegraphics[width=0.9\textwidth]{images/konzept_cache.pdf}}
\caption{Caching Strategie}
\label{image_konzept_caching-strategie}
\end{figure} 

\subsection{Caching des anwendungspezifischen Modells}
\label{subsec_konzeption_caching-modell}

Neben der in Abschnitt \ref{subsec_konzept_caching-statische-ressourcen} beschriebenen Vorhaltung der statischen Ressourcen muss die hybride Webanwendung ebenfalls einen Mechanismus zur Verfügung stellen, um das Datenmodell im Offlinebetrieb bereitzustellen.

\newpage
\section{Web Push}
\label{subsubsec_konzeption_serviceworker_push-api}

\begin{figure}[htp] \centering{
\centering
\includegraphics[width=0.9\textwidth]{images/architektur_serviceworker_push.pdf}}
\caption{Push mittels Serviceworker (in Anlehnung an MozillaWiki \cite{MOZ_WIKI})}
\quelle\url{https://wiki.mozilla.org/File:PushNotificationsHighLevel.png}
\label{image_architektur-serviceworker-push}
\end{figure}  


\todo{Beschreibung (mit Schema) der Softwarearchitektur ...}

\newpage
\section{Architekturbeschreibung}
\label{sec_konzeption_serviceworker_architektur}

Die Anwendung beruht auf dem Client-Server-Prinzip. Dabei stellt der Client lediglich die Oberfläche zur Interaktion mit dem Anwender dar. Außer der notwendigen UI- und Serviceworker-Logik ist die gesamte Geschäftslogik auf einen Business-Server (Applicationserver) ausgelagert. Die zentrale Datenbank sowie die statischen Ressourcen zur Darstellung des Client werden ebenfalls vom Applicationserver bereitgestellt. Für die Kommunikation steht eine RESTful-Schnittstelle zur Verfügung.

\begin{figure}[htp] \centering{
\includegraphics[width=0.9\textwidth]{images/architektur_serviceworker.pdf}}
\caption{Archtikturbeschreibung - Umsetzung mit Serviceworker}
\label{image_architektur-serviceworker-push}
\end{figure} 

\newpage
\section{Applicationserver}
\label{sec_konzeption_applicationserver}

\subsection{Datenbank}



\subsection{REST-API}

Zur Bereitstellung von CRUD-Funktionalitäten über standardisierte HTTP-Methoden (vgl. \ref{subsec_anforderungen_server} Anforderungen an Serverkomponente) wird dem Applicationserver eine RESTful-Schnittstelle hinzugefügt. Eine Übersicht über mögliche API-Routen mit entsprechender HTTP-Methode ist in Tabelle \ref{tbl_konzeption_rest} dargestellt. \\

\begin{table}[h]
\centering
\begin{tabular}{l | c | l }
    \textbf{Route} & \textbf{HTTP-Methode} & \textbf{Beschreibung} \\
    \hline\hline
    /api/signup & POST & Registriert einen neuen Benutzer \\
    /api/authenticate & POST & Authentifiziert einen Benutzer \\
    \hline
    /api/tasks & GET & Gibt alle Aufgaben zurück \\
    /api/tasks & POST & Legt eine neue Aufgabe an \\
    /api/tasks/:taskId & GET & Gibt eine einzelne Aufgabe zurück \\
    /api/tasks/:taskId & PUT & Aktualisiert eine einzelne Aufgabe \\
    /api/tasks/:taskId & DELETE & Löscht eine einzelne Aufgabe \\
    \hline
    /push/devices & GET & Gibt alle registrierten Geräte zurück \\
    /push/devices & POST & Registriert ein neues Gerät \\
\end{tabular}
\caption{Übersicht API Routen}
\label{tbl_konzeption_rest}
\end{table}

\subsubsection{Authentifizierung und Autorisierung} 

Für den Zugriff auf die CRUD-Methoden ist eine Benutzerauthentifizierung und Autorisierung notwendig. Dazu wird das Token-Verfahren verwendet.

\newpage
\section{Client-Oberfläche}
\label{sec_konzeption_client-ui}


... Mockup ... \\
... Beschreibung des UI ...\\

Um das \glqq{}Look and Feel\grqq{} einer nativen App zu erreichen wird das UI-Framework \textbf{nativeDroid2} verwendet. \\

\newpage
\section{Datenmodel}
\label{sec_konzeption_datamodel}

\begin{figure}[htp] \centering{
\includegraphics[width=0.9\textwidth]{images/model.png}}
\caption{Datenmodell}
\label{image_konzeption_datenmodell}
\end{figure} 

\newpage

\chapter{Implementierung}

Das folgende Kapitel beschreibt die Umsetzung der im vorherigen Kapitel beschriebenen Konzeption. Dabei wird sich sowohl bei der serverseitigen als auch clientseitigen Implementierung auf die \todo{anderes Wort?}[Bereiche] konzentriert, die für die Verwendung der Service Worker Technologie und die Erfüllung der Projektaufgabe notwendig sind. Die Umsetzung der grafischen Benutzeroberfläche ist damit nicht Bestandteil dieser Dokumentation.

\section{Applicationserver}

Neben einer Datenbank bietet der Applicationserver die Funktionalitäten eines Webservers. Als Backend"=Plattform wird Node.js eingesetzt und mit Hilfe von \code{Node Express} ist in wenigen Schritten ein voll-funktionsfähiger Webserver installiert und eingerichtet.  

\begin{lstlisting}[caption=package.json - notwendige Node.js Pakete, frame=single]
{
  "name": "node-rest-auth",
  "main": "server.js",
  "dependencies": {
    "bcrypt": "^0.8.5",
    "body-parser": "~1.9.2",
    "express": "~4.9.8",
    "jwt-simple": "^0.3.1",
    "mongoose": "~4.2.4",
    "morgan": "~1.5.0",
    "passport": "^0.3.0",
    "passport-jwt": "^1.2.1"
  }
}
\end{lstlisting}


\subsection{Datenbank}
\label{subsec_implementierung_datenbank}

\subsection{REST-Schnittstelle}

\section{Serviceworker}

\subsection{Caching der statischen Ressourcen}

\subsection{Persistenz und Synchronisation des Application State}


\todo{Hier noch etwas über die Background.Sync API sagen\ldots}

\todo{Eventuell passt hier die ALternative PouchDB hin\ldots}


\subsection{Push-Notification}


\begin{lstlisting}
// /js/app.js
if ('serviceWorker' in navigator)
{
    navigator.serviceWorker.register('sw.js').then(function(reg) {

        if(reg.installing)
        {
            console.log('Service worker installing');
        } 
        else if(reg.waiting)
        {
            console.log('Service worker installed');
        } 
        else if(reg.active)
        {
            console.log('Service worker active');
        }

    }).catch(function(error)
    {
        // registration failed
        console.log('Registration failed with ' + error);
    });
}

\caption{Einrichtung Serice Worker}
\end{lstlisting}

\newpage 

\chapter{Zusammenfassung und Ausblick}

... Was kann nicht geleistet werden? ...

... Was ist eventuell zukünftig möglich ? ...

\section{Background Sync }

We're aiming to ship background sync to a stable version of Chrome in the first half of 2016. But we’re also working on a variant, “periodic background sync”. This will allow you to request a “periodicsync” event restricted by time interval, battery state and network state. This would require user permission, of course, but it will also be down to the will of the browser for when and how often these events fire. E.g., a news site could request to sync every hour, but the browser may know you only read that site at 07:00, so the sync would fire daily at 06:50. This idea is a little further off than one-off syncing, but it’s coming.

Bit by bit we’re bringing successful patterns from Android/iOS onto the web, while still retaining what makes the web great!


\section{Fazit}

Offlinefähigkeit ist ein sehr interessantes Feature, das die Benutzererfahrung mit mobilen Anwendungen erheblich verbessern kann. Wenn der Anwender ständigen Verbindungsabbrüchen ausgesetzt ist, scheint diese Funktionalität ein unabdingbares Element der modernen Entwicklung von mobilen Apps zu sein.

Für die Art der Umsetzung muss man sich entsprechend Gedanken machen, ob die einfachste Variante reicht oder ob das Businessszenario die echte Offlinefähigkeit notwendig macht. Der Fokus muss hierbei auf der Erkennung der Bedürfnisse des Kunden liegen, um die passende Variante zu wählen.

Am Ende unserer Projektarbeit möchten wir folgendes Fazit ziehen, die Service Worker-Technologie ist eine mächtige Technologie, die es erlaubt Funktionalitäten zur Verfügung stellen, die unabhängig von einer Webseite oder Benutzerinteraktion sind. Dazu gehören zum Beispiel eine Cache-Funktion, die es ermöglicht, einmal angezeigte Inhalte in den Speicher(Cache) zu laden. Auf diese Weise kann beim nächsten Besuch die Seite auch dann angezeigt werden, wenn eine schlechte oder sogar gar keine Internetverbindung besteht (Offline-Betrieb). \\
Weiterhin sind Push-Benachrichtigungen wie bei nativen Apps möglich, um Benutzer auf neue Ereignisse hinzuweisen. Service Worker müssen im JavaScript der Seite registriert werden und können erst dann installiert werden und bedingen HTTPS. 

Wir sind bei der Implementierung und Konfiguration auf einige kleine Probleme gestoßen, die sich aber letztendlich alle bewerkstelligen ließen.  


%Literaturverzeichnis
\newpage
\bibliographystyle{unsrtdin}
%\bibliographystyle{gerplain}
\bibliography{Literatur}
\thispagestyle{fancy}

\newpage
% Abbildungsverzeichnis
\listoffigures

% ANHANG
\newpage
\addtocontents{toc}{\protect\value{tocdepth}=-1}%
\captionsetup{list=false} 
%Seitennummerierung neu beginnen, Zahlen [arabic], röm.Zahlen [roman,Roman], Buchstaben [alph,Alph]
\pagenumbering{Roman}
\chapter{Anhang}
%\section{API Beschreibung}

\begin{table}[h]
    \begin{tabular}{ l }
        Löscht eine einzelne Aufgabe \\
    \end{tabular}
    \caption{Übersicht API Routen}
    \label{tbl_api-routes}
\end{table}



{\Large \textbf{/api/signup}}


\fbox{\parbox{0.5\linewidth}{
    \textbf{Request:}
    \begin{itemize}  
        \item {\tshortstack[l]{\textbf{\code{username}}: (String)\\gewünschter Benutzername}}
        \item {\tshortstack[l]{\textbf{\code{password}}: (String)\\Passwort}}
        \item {\tshortstack[l]{\textbf{\code{email}}: (String)\\E-Mail Adresse}}
    \end{itemize}    
}
\parbox{0.5\linewidth}{
    \textbf{Response:}
    \begin{itemize}  
        \item {\tshortstack[l]{\textbf{\code{username}}: (String)\\gewünschter Benutzername}}
        \item {\tshortstack[l]{\textbf{\code{password}}: (String)\\Passwort}}
        \item {\tshortstack[l]{\textbf{\code{email}}: (String)\\E-Mail Adresse}}
    \end{itemize}    
}}

\bgroup
\def\arraystretch{1}%  1 is the default, change whatever you need

\begin{tabularx}{\textwidth}{|l|X|}
    \hline
    \multicolumn{2}{|c|}{Benutzer anlegen } \\
    \hline
    URL & \textbf{\code{/api/signup}} \\
    \hline
    Methode & \code{GET} \\
    \hline
    Request-Parameter & Required: 
      \begin{itemize}  
        \item {\tshortstack[l]{\textbf{\code{username}}: (String)\\gewünschter Benutzername}}
        \item {\tshortstack[l]{\textbf{\code{password}}: (String)\\Passwort}}
        \item {\tshortstack[l]{\textbf{\code{email}}: (String)\\E-Mail Adresse}}
      \end{itemize} \\
    \hline
    Success-Response & 
      \begin{itemize}  
        \item {Code: 200}
        \item {Content: \code{{success: true, message: 'Successful created new user.'}}}
      \end{itemize} \\
    \hline
    \multicolumn{2}{|l|}{Legt einen neuen Benutzer an. } \\
    \hline
    \multicolumn{2}{|l|}{Request } \\
    \multicolumn{2}{|l|}{ \textbf{\code{username}}: (String) gewünschter Benutzername} \\
    \multicolumn{2}{|l|}{ \textbf{\code{password}}: (String) Passwort} \\
    \multicolumn{2}{|l|}{ \textbf{\code{email}}: (String) E-Mail Adresse} \\
    \hline
\end{tabularx}
\egroup
        
\begin{itemize}  
        \item {\tshortstack[l]{\textbf{\code{username}}: (String)\\gewünschter Benutzername}}
        \item {\tshortstack[l]{\textbf{\code{password}}: (String)\\Passwort}}
        \item {\tshortstack[l]{\textbf{\code{email}}: (String)\\E-Mail Adresse}}
    \end{itemize}  
    \textbf{Response} 
    \begin{itemize}  
        \item Bereitstellung von CRUD\footnote{\textit{CRUD: create, read, update, delete}}-Funktionalität für Entities
        \item Aufruf von Ressourcen über eindeutige und einfache URLs (z.B. https://example.de/api/task/ und https://example.de/api/task/:taskId) 
    \end{itemize} 
%\newpage
\section{Applicationserver}
\subsection{mongoos-Schema User-Entity}
\label{subsub_a2_mongoose-schema}

\begin{lstlisting}[caption={Mongoose Schema der User-Entity},label={lst_a2_model-user}, frame=single]
/**
 * USER Entity
 */

// load ORM
var mongoose = require('mongoose');
var Schema = mongoose.Schema;
var bcrypt = require('bcrypt-nodejs');

// create user schema
var UserSchema = new Schema({
    name: String,
    username: {
        type: String,
        unique: true,
        required: true
    },
    email: {
        type:String,
        unique: true,
        required: true
    },
    password: {
        type: String,
        required: true
    },
    admin: Boolean,
    created_at: Date,
    updated_at: Date
});

// on every save set updated_at and salt password
UserSchema.pre('save', function (next) {
    var user = this;
    var currentDate = new Date();

    // set updated_at date
    user.updated_at = currentDate;
    // if created_at doesn't exist, add to field
    if(!user.created_at)
        user.created_at = currentDate;

    // save salted password to database if modified
    if (this.isModified('password') || this.isNew) {
        bcrypt.genSalt(10, function (err, salt) {
            if (err) {
                return next(err);
            }
            bcrypt.hash(user.password, salt, null, function (err, hash) {
                if (err) {
                    return next(err);
                }
                user.password = hash;
                next();
            });
        });
    } else {
        return next();
    }
});

// compare salted passwords
UserSchema.methods.comparePassword = function (passwd, cb) {
    bcrypt.compare(passwd, this.password, function (err, isMatch) {
        if (err) {
            return cb(err);
        }
        cb(null, isMatch);
    });
};

module.exports = mongoose.model('User', UserSchema);
\end{lstlisting}

\newpage
\subsection{Router"=Middleware zur absicherung der API}
\label{subsub_a2_api-security}


\begin{lstlisting}[caption={Mongoose Schema der User-Entity},label={lst_a2_model-user}, frame=single]
// routes/api.js

/**
 * SECURITY: middleware to protect API
 **********************************************/
getToken = function (headers) {
    if (headers && headers.authorization) {
        return headers.authorization;
        var parted = headers.authorization.split(' ');
        if (parted.length === 2) {
            return parted[1];
        } else {
            return null;
        }
    } else {
        return null;
    }
};

router.use(function(req, res, next) {
    console.log(req.headers);
    // check header or url parameters or post parameters for token
    var token = req.body.token || req.query.token || req.headers['x-access-token'] || getToken(req.headers);

    // decode token
    if (token) {
        // verifies secret and checks exp
        jwt.verify(token, config.secret, function(err, decoded) {
            if (err) {
                return res.status(403).json({ success: false, message: 'Failed to authenticate token.' });
            } else {
                // if everything is good, save to request for use in other routes
                User.findOne({
                    username: decoded.username
                }, function(err, user) {
                    if (err) throw err;

                    if (!user) {
                        return res.status(403).send({success: false, message: 'Authentication failed. User not found.'});
                    } else {
                        req.user = user;
                        next();
                    }
                });
            }
        });
    } else {
        // if there is no token return an error
        return res.status(403).send({
            success: false,
            message: 'No token provided.'
        });
    }
});
\end{lstlisting}

\newpage
\section{Service Worker Konfiguration}
\subsection{Ressourcen für Caching festlegen}
\label{subsec_a3_caching}

\begin{lstlisting}[caption={Service Worker Konfiguration - Ressourcen für Caching},label={lst_a3_sw-caching}, frame=single]
var urlsToCache = [
    './vendor/nativedroid2/css/nativedroid2.color.blue-grey.css',
    './vendor/nativedroid2/css/nativedroid2.color.teal.css',
    './vendor/nativedroid2/css/flexboxgrid.min.css',
    './vendor/nativedroid2/css/material-design-iconic-font.min.css',
    './vendor/nativedroid2/fonts/Material-Design-Iconic-Font.eot',
    './vendor/nativedroid2/fonts/Material-Design-Iconic-Font.svg',
    './vendor/nativedroid2/fonts/Material-Design-Iconic-Font.ttf',
    './vendor/nativedroid2/fonts/Material-Design-Iconic-Font.woff',
    './vendor/nativedroid2/js/nativedroid2.js',
    './vendor/fingerprint2js/fingerprint2.js',

    './vendor/font-awesome/css/font-awesome.min.css',
    './vendor/font-awesome/fonts/FontAwesome.otf',
    './vendor/font-awesome/fonts/fontawesome-webfont.eot',
    './vendor/font-awesome/fonts/fontawesome-webfont.svg',
    './vendor/font-awesome/fonts/fontawesome-webfont.ttf?v=4.6.3',
    './vendor/font-awesome/fonts/fontawesome-webfont.woff?v=4.6.3',
    './vendor/font-awesome/fonts/fontawesome-webfont.woff2?v=4.6.3',

    './vendor/jquery/jquery-3.1.1.min.js',
    './vendor/jquery/jquery-migrate-3.0.0.js',
    './vendor/jquery-mobile/jquery.mobile-1.4.5.min.js',
    './vendor/jquery-mobile/jquery.mobile-1.4.5.min.css',
    './vendor/jquery-mobile/images/ajax-loader.gif',
    './vendor/jquery-ui/jquery-ui.min.js',
    './vendor/jquery-ui/jquery-ui.min.css',
    './vendor/jquery-validate/jquery.validate.min.js',

    './vendor/waves/waves.min.js',
    './vendor/waves/waves.min.js.map',
    './vendor/waves/waves.min.css',
    './vendor/wow/animate.css',
    './vendor/wow/wow.min.js',

    './vendor/idb/',
    './vendor/idb/lib/',
    './vendor/idb/lib/idb.js',

    './config/nd2settings.js',
    './fragments/bottom.sheet.html',
    './fragments/panel.left.html',
    './fragments/page.home.html',
    './fragments/page.login.html',
    './fragments/page.register.html',
    './fragments/page.task.add.html',

    './resources/css/style.css',
    './resources/fonts/Roboto-Regular.ttf',
    './resources/img/2.jpg',
    './resources/img/8.jpg',
    './resources/img/9.jpg',
    './resources/img/10.jpg',
    './resources/img/examples/card_bg_1.jpg',
    './resources/img/examples/card_bg_2.jpg',
    './resources/img/examples/card_bg_3.jpg',
    './resources/img/examples/card_thumb_1.jpg',
    './resources/img/examples/card_thumb_2.jpg',
    './resources/img/examples/card_thumb_3.jpg',

    './resources/js/app.js',
    './resources/js/pushFunctions.js',
    './resources/js/validation.js',
    './resources/js/home.js',
    './manifest.json',
    './index.php',
    './'
];
\end{lstlisting}

\newpage
\section{PushFunctions.js}
\label{sec_a3_sw-push-functions}


\begin{lstlisting}[caption={PushFunctions.js - Verarbeitung der PushSubscriptions}, frame=single]
// resources/js/pushFunctions.js

var PUSH_URL = "http://localhost:3000/push";

/**
 * Subscribe push
 *
 * - creates push manager subscription
 * - sends subscription to application server
 */
function subscribePush() {
	  navigator.serviceWorker.ready.then(function(serviceWorkerRegistration) {
	    serviceWorkerRegistration.pushManager.subscribe({userVisibleOnly: true})
	      .then(function(pushSubscription) {
	          //Store this subscription on application server
              sendSub(pushSubscription);
            return true;
	      })
	      .catch(function(e) {
	        console.error('Unable to register push subscription', e);
	        return false;
	      });
	  });
	}

/**
 * unsubscribe push
 *
 * - unsubscribe push manager
 * - remove subscription from application server
 */
function unsubscribePush() {
	  console.log('unsubscribing...');
	  navigator.serviceWorker.ready.then(function(serviceWorkerRegistration) {

	  serviceWorkerRegistration.pushManager.getSubscription()
		.then(
		  function(pushSubscription) {
			// We have a subscription, so remove it from applications server...
			cancelSub(pushSubscription);
			//... and unsubscribe it
			pushSubscription.unsubscribe().then(function() {}).catch(function(e) {
			  console.log('Error unsubscribing: ', e);
			});
	     })
		.catch(function(e) {
	      	console.error('Error unsubscribing.', e);
	   	});
	});  
}

/**
 * send Subscription to application server
 */
function sendSub(pushSubscription) {
	var deviceId = localStorage.getItem('deviceId');
	var deviceName = "Hier wird iwann der Username stehen";
    var endpoint = pushSubscription.endpoint;
    var subId = endpoint.split("/").pop();

    localStorage.setItem('gcm-regid', subId);

    var authToken = localStorage.getItem("auth-token");
    fetch(PUSH_URL + "/devices/", {
        mode: 'cors',
        method: 'POST',
        headers: {
            "Content-Type": "application/x-www-form-urlencoded",
	    },
        body: "deviceName="+deviceName+"&deviceId="+deviceId+"&registrationId="+subId+"&endpoint="+endpoint+"&token="+authToken,
    })
		.then(function(res) {
		  res.json().then(function(data) {
              // Log the data for illustration
              console.log(data);
        });
  });
}
\end{lstlisting}

\end{document}