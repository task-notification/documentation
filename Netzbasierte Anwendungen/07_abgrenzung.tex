\chapter{Zusammenfassung und Ausblick}

... Was kann nicht geleistet werden? ...

... Was ist eventuell zukünftig möglich ? ...

\section{Fazit}

Offlinefähigkeit ist ein sehr interessantes Feature, das die Benutzererfahrung mit mobilen Anwendungen erheblich verbessern kann. Wenn der Anwender ständigen Verbindungsabbrüchen ausgesetzt ist, scheint diese Funktionalität ein unabdingbares Element der modernen Entwicklung von mobilen Apps zu sein.

Für die Art der Umsetzung muss man sich entsprechend Gedanken machen, ob die einfachste Variante reicht oder ob das Businessszenario die echte Offlinefähigkeit notwendig macht. Der Fokus muss hierbei auf der Erkennung der Bedürfnisse des Kunden liegen, um die passende Variante zu wählen.