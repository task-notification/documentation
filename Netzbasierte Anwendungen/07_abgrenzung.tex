\chapter{Ausblick}

Offlinefähigkeit ist ein sehr interessantes Feature, das die Benutzererfahrung mit mobilen Anwendungen erheblich verbessern kann. Wenn der Anwender ständigen Verbindungsabbrüchen ausgesetzt ist, scheint diese Funktionalität ein unabdingbares Element der modernen Entwicklung von mobilen Apps zu sein.

Für die Art der Umsetzung muss man sich entsprechend Gedanken machen, ob die einfachste Variante reicht oder ob das Businessszenario die echte Offlinefähigkeit notwendig macht. Der Fokus muss hierbei auf der Erkennung der Bedürfnisse des Kunden liegen, um die passende Variante zu wählen.

Die Service Worker API zielt genau auf die Lücke, die zwischen mobile Web Apps und nativen Apps existiert. Besonders die Bereitstellung von Konzepten und Methoden zur Unterstützung einer Offlinefähigkeit, Abarbeitung von Hintergrundaktivitäten und die Möglichkeit der Verwendung Push"=Nachrichten im inaktiven Modus stellen einen großen Schritt bei der Annäherung von mobile Webseiten an native Apps dar.

Es ist zu erwarten, dass sich die Service Worker Technologie weiter durchsetzen wird und damit weiter verbreitet. Besonders die Unterstützung mobiler Browser wird dabei im Mittelpunkt stehen. 