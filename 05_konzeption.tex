\chapter{Konzeption}
\label{sec_konzeption}

\section{Umsetzung mittels Serviceworker}
\label{subsec_konzeption_serviceworker}

\subsection{Architekturbeschreibung}
\label{subsubsec_konzeption_serviceworker_architektur}

Die Anwendung beruht auf dem Client-Server-Prinzip. Dabei stellt der Client lediglich die Oberfläche zur Interaktion mit dem Anwender dar. Außer der notwendigen UI Logik ist die gesamte Geschäftslogik auf einen dedizierten (Business-)Server ausgelagert. Dieser stellt ebenfalls die Datenbank bereit. Mittels AJAX und REST werden dynamische Daten vom Client beim Server angefragt.

\begin{figure}[htp] \centering{
\includegraphics[width=0.9\textwidth]{images/architektur_serviceworker.pdf}}
\caption{Archtikturbeschreibung - Umsetzung mit Serviceworker}
\label{image_architektur-serviceworker-push}
\end{figure} 

\newpage
\subsection{Push API}
\label{subsubsec_konzeption_serviceworker_push-api}

\begin{figure}[htp] \centering{
\centering
\includegraphics[width=0.9\textwidth]{images/architektur_serviceworker_push.pdf}}
\caption{Push mittels Serviceworker (in Anlehnung an MozillaWiki \cite{MOZ_WIKI})}
\quelle\url{https://wiki.mozilla.org/File:PushNotificationsHighLevel.png}
\label{image_architektur-serviceworker-push}
\end{figure}  

... Beschreibung (mit Schema) der Softwarearchitektur ...

\newpage
\section{Umsetzung mittels nativer Android Dienst}
\label{subsec_konzeption_android}

\subsection{Architekturbeschreibung}
\label{subsubsec_konzeption_android_architektur}

\begin{figure}[htp] \centering{
\includegraphics[width=0.9\textwidth]{images/architektur_android.pdf}}
\caption{Architektur - Umsetzung mit nativem Android Dienst }
\label{image_architektur-android-push}
\end{figure}  

... Beschreibung (mit Schema) der Softwarearchitektur ...

\newpage
\subsection{Push}

\begin{figure}[htp] \centering{
\includegraphics[width=0.9\textwidth]{images/architektur_android_push.pdf}}
\caption{Push mittels nativem Android Dienst (in Anlehnung an MozillaWiki \cite{MOZ_WIKI})}
\quelle\url{https://wiki.mozilla.org/File:PushNotificationsHighLevel.png}
\label{image_architektur-android-push}
\end{figure}  

\section{Datenmodell}
\label{subsec_datenmodell}

\begin{figure}[htp] \centering{
\includegraphics[width=0.9\textwidth]{images/model.png}}
\caption{Datenmodell}
\label{image_konzeption_datenmodell}
\end{figure} 


\section{Serverkomponente}
\label{subsec_serverkomponente}

\subsection{REST-API}

Zur Bereitstellung von CRUD-Funktionalitäten über standardisierte HTTP-Methoden (vgl. \ref{subsec_anforderungen_server} Anforderungen an Serverkomponente) wird dem Applicationserver eine RESTful-Schnittstelle hinzugefügt. Eine Übersicht über mögliche API-Routen mit entsprechender HTTP-Methode ist in Tabelle \ref{tbl_api-routes} dargestellt. \\

\begin{table}[h]
\centering
\begin{tabular}{l | c | l }
    \textbf{Route} & \textbf{HTTP-Methode} & \textbf{Beschreibung} \\
    \hline\hline
    /api/signup & POST & Registriert einen neuen Benutzer \\
    /api/authenticate & POST & Authentifiziert einen Benutzer \\
    \hline
    /api/tasks & GET & Gibt alle Aufgaben zurück \\
    /api/tasks & POST & Legt eine neue Aufgabe an \\
    /api/tasks/:taskId & GET & Gibt eine einzelne Aufgabe zurück \\
    /api/tasks/:taskId & PUT & Aktualisiert eine einzelne Aufgabe \\
    /api/tasks/:taskId & DELETE & Löscht eine einzelne Aufgabe \\
\end{tabular}
\caption{Übersicht API Routen}
\label{tbl_api-routes}
\end{table}

\subsubsection{Authentifizierung und Autorisierung} 

Für den Zugriff auf die CRUD-Methoden ist eine Benutzerauthentifizierung und Autorisierung notwendig. Dazu wird das Token-Verfahren verwendet.


\section{Benutzeroberfläche}

... Mockup ... \\
... Beschreibung des UI ...\\

Um das \glqq{}Look and Feel\grqq{} einer nativen App zu erreichen wird das UI-Framework \textbf{nativeDroid2} verwendet. \\
