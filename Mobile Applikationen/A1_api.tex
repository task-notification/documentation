\section{API Beschreibung}

\begin{table}[h]
    \begin{tabular}{ l }
        Löscht eine einzelne Aufgabe \\
    \end{tabular}
    \caption{Übersicht API Routen}
    \label{tbl_api-routes}
\end{table}



{\Large \textbf{/api/signup}}


\fbox{\parbox{0.5\linewidth}{
    \textbf{Request:}
    \begin{itemize}  
        \item {\tshortstack[l]{\textbf{\code{username}}: (String)\\gewünschter Benutzername}}
        \item {\tshortstack[l]{\textbf{\code{password}}: (String)\\Passwort}}
        \item {\tshortstack[l]{\textbf{\code{email}}: (String)\\E-Mail Adresse}}
    \end{itemize}    
}
\parbox{0.5\linewidth}{
    \textbf{Response:}
    \begin{itemize}  
        \item {\tshortstack[l]{\textbf{\code{username}}: (String)\\gewünschter Benutzername}}
        \item {\tshortstack[l]{\textbf{\code{password}}: (String)\\Passwort}}
        \item {\tshortstack[l]{\textbf{\code{email}}: (String)\\E-Mail Adresse}}
    \end{itemize}    
}}

\bgroup
\def\arraystretch{1}%  1 is the default, change whatever you need

\begin{tabularx}{\textwidth}{|l|X|}
    \hline
    \multicolumn{2}{|c|}{Benutzer anlegen } \\
    \hline
    URL & \textbf{\code{/api/signup}} \\
    \hline
    Methode & \code{GET} \\
    \hline
    Request-Parameter & Required: 
      \begin{itemize}  
        \item {\tshortstack[l]{\textbf{\code{username}}: (String)\\gewünschter Benutzername}}
        \item {\tshortstack[l]{\textbf{\code{password}}: (String)\\Passwort}}
        \item {\tshortstack[l]{\textbf{\code{email}}: (String)\\E-Mail Adresse}}
      \end{itemize} \\
    \hline
    Success-Response & 
      \begin{itemize}  
        \item {Code: 200}
        \item {Content: \code{{success: true, message: 'Successful created new user.'}}}
      \end{itemize} \\
    \hline
    \multicolumn{2}{|l|}{Legt einen neuen Benutzer an. } \\
    \hline
    \multicolumn{2}{|l|}{Request } \\
    \multicolumn{2}{|l|}{ \textbf{\code{username}}: (String) gewünschter Benutzername} \\
    \multicolumn{2}{|l|}{ \textbf{\code{password}}: (String) Passwort} \\
    \multicolumn{2}{|l|}{ \textbf{\code{email}}: (String) E-Mail Adresse} \\
    \hline
\end{tabularx}
\egroup
        
\begin{itemize}  
        \item {\tshortstack[l]{\textbf{\code{username}}: (String)\\gewünschter Benutzername}}
        \item {\tshortstack[l]{\textbf{\code{password}}: (String)\\Passwort}}
        \item {\tshortstack[l]{\textbf{\code{email}}: (String)\\E-Mail Adresse}}
    \end{itemize}  
    \textbf{Response} 
    \begin{itemize}  
        \item Bereitstellung von CRUD\footnote{\textit{CRUD: create, read, update, delete}}-Funktionalität für Entities
        \item Aufruf von Ressourcen über eindeutige und einfache URLs (z.B. https://example.de/api/task/ und https://example.de/api/task/:taskId) 
    \end{itemize} 